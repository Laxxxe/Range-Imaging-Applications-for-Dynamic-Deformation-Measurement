\section[Introduction]{Introduction}\label{sec:motivation}
The interest in new innovative kinds of human/computer interaction lead to multiple low cost developments to detect the movement of a human body. The Kinect 2, developed by Microsoft, measures the propagation time of light by the Time-of-Flight (ToF) principle, to derive 3D data from the observed scene. The field of view is illuminated by IR didoes and a IR CMOS camera measures the time that the light take to reflect back into the camera. With the constant speed of light and the measured time of flight the distance, that the light traveled, can be calculated.  Since the appearance of ToF cameras in the year 2000, the pixel resolution, dynamic range and the accuracy have significantly improved. In contrast to stereo triangulation between multiple cameras or 3D scanning with a moving laser beam, this technique allows low cost and small size applications for 3D scanning. High frame rates can be achieved and the 3D data is acquired on every pixel at the same time.\\

 The acquisition of real-time 3D data also offers various applications in experimental setups. In crash test facility for example the geometry and orientation of the structures before and after the test can be measured. In windtunnel applications, the angle of attack and vibrations of the aerodynamic body can be observed by the camera without disturbing markers on the surfaces. The measurement of the static geometries and the dynamic behavior delivers a way to merge data between different calibrated camera systems in a common set of data. Aeroelasticity investigations are possible on the complete body in every point of the field of view.\\

 This thesis will reveal how static and dynamic ToF data can be investigated and processed. Xiao Qi and Derek Lichti already showed the performance of dynamic beam deformation measurement with ToF \cite{qi2014structural}. A $1~Hz$ sinus oscillation of an concrete beam with an amplitude $4~mm$ was reconstructed with an accuracy of $0.061~mm$. This was possible with two phase synchronized ToF cameras in a stereo arrangement for a single point on the beam. The noise and uncertainty between both sensors are canceled out between both measurements. The advantages of static investigation with a Time-of-Flight camera for structural deformation measurement have already been proven with last generation devices \cite{jamtsho2010geometric}. In previous investigations of 3D scanning techniques for static deformation, Terrestrial Laser Scanning also showed a high performance, but the systems are slow, expensive and large \cite{park2007new}.\\
    
 Advanced ToF image sensors with a higher pixel resolution, accuracy and improved noise characteristic promise the dynamic investigations of complete surfaces with a single camera. A depth resolution up to $0.3~mm$ is possible, with current CMOS technology \cite{yasutomi20147}. Other devices feature high frame rates of $470~fps$ in a 1280x1024 pixel area \cite{odosimagingdatasheet}. The dynamic performance of the Kinect 2 ToF camera, that is originally designed for human/computer interaction, will be investigated. The accuracy of the measurement will be compared to a Laser Doppler Vibrometer (LDV). A speaker membrane delivers a constant dynamic movement with a calibrated sinus function as input. Therefore, deformation measurement like Stereo Pattern Tracking can be investigated on a controlled test body \cite{mantik2013enhancement}. Boundary conditions like surface material, geometry and angle of the camera to the surface normal will be analyzed in terms of precision and accuracy. Another major advantage of range data is the simple image processing that will be shown in Matlab for some basic examples for the static and dynamic structure investigation. The creation of surface dynamics images, showing the frequency and amplitude distribution, is one of the main subjects in this context. This delivers a simple and easy way to obtain the dynamic behavior of complete surfaces. Applications of Time-of-Flight cameras are shown in multiple small experiments. The revealed procedures are only supposed to show the performance of the measurement principle in a rough way and demonstrate how a real calibration of the ToF System in a controlled environment can look like. ToF cameras have not been investigated previously at the Department of Aerospace Technology which offers the possibility to build up knowledge from a unprejudiced basis. The source code for image processing has been developed without any examples. Improvements are still necessary in the experimental setup, source code and suggested theories.     