Three-dimensional Time-of-Flight (ToF) range cameras improved significantly in accuracy, speed, price and size over the last years. By delivering the range from the scene to the image sensor in every pixel, static and dynamic scenes can be reconstructed into a three-dimensional model contactless from a single point of view. The observed area is illuminated by infrared diodes and a IR CMOS camera measures the time delay of the returning light \cite{thierryoggietof}. Out of it the range is acquired parallel with no time delay between the pixels \cite{litime}. This offers the potential for various close to medium range applications. One of this is the measurement of vibrations and structural deformations under dynamic loads \cite{qi2014structural}. In contrast to other present measurement methods, complete surface movements can be acquired in parallel without a movable laser beam or a complex stereo triangulation between multiple cameras . The Kinect 2 range camera is used to show the performance of vibration measurements and to analyze how the range images can be filtered and processed. It is the first ToF camera on the consumer market and at the time of release the one with the highest number of pixels ($512x424$ at 30 frames per seconds). It delivers 3D informations from $0.6~m$ to $8.0~m$ at a depth resolution of $1~mm$ \cite{sell2014xbox}. A Laser Doppler Vibrometer is used, as high precision reference measurement device, delivering the highest standard in deformation measurement on a single surface point. This thesis will prove, that vibrations of a body can be reconstructed at amplitudes of $4~mm$ with an accuracy below $1~mm$. The performance depends on the distance to the camera, angle of view, surface material, number of investigated pixels and the measurement time. At the ideal distance of $1.3~m$ to $1.5~m$, the error to the reference is reduced down to $0.05~mm$ after 8 seconds of measurement at a frequency of $3~Hz$. Even the first and second harmonics are visible in the frequency domain on white paper, offering a high reflectivity for the infrared light of the ToF measurement principle. The influence on the accuracy of different angles to the surface normal is another topic. A vibration of $3~Hz$ can be reconstructed correctly in frequency up to an angle of $45^\circ$. The amplitude is overestimated with higher angles. Since every pixel delivers a range information, the complete video stream can be processed by a Fast-Fourier Transform to derive the dynamic behavior of the whole field of view. The vibration can be illustrated in images, representing the amplitude and frequency distribution in the complete field of view over a certain time span.\\

This development offers new applications in the static and dynamic observation and investigation of machines and structures. Positions, geometries, movements, orientations and other object properties can be analyzed and reconstructed in real time. Industrial installations and processes such as engines can be observed and the production quality can be improved.